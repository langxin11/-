%!TEX program = xelatex

%%%%%%%%%%%%%%%%%%%%%%%%%%%%%%%%%%%%%%%%%%%%%%%%%%%%%%%%
% 文件名:main.tex
% 编译器:XeLaTeX (必须)
% 字体需求:需在根目录创建 fonts 文件夹并上传 simsun.ttc, times.ttf, calibri.ttf
% 参考文献:需在根目录创建 ref.bib 并填入 BibTeX 数据
%%%%%%%%%%%%%%%%%%%%%%%%%%%%%%%%%%%%%%%%%%%%%%%%%%%%%%%%

\documentclass[a4paper,zihao=-4]{ctexart} 

%=================== 1. 宏包加载 ===================
\usepackage{geometry}       
\usepackage{fancyhdr}       
\usepackage{fontspec}       
\usepackage{titlesec}       
\usepackage{titletoc}       
\usepackage{setspace}       
\usepackage{amsmath}        
\usepackage{graphicx}       
\usepackage{float}          
\usepackage{cite}           
\usepackage{caption}        
\usepackage{anyfontsize}
\usepackage{xcolor}         % 用于TODO高亮

% 【修复红框】加载超链接包并隐藏边框
\usepackage[hidelinks]{hyperref}

% 定义 TODO 命令,红色加粗显示,用于提示组员
\newcommand{\todo}[1]{\textcolor{red}{\textbf{[TODO: #1]}}}

% 让每个一级标题自动换页
\newcommand{\sectionbreak}{\clearpage}

%=================== 2. 页面布局与行距 ===================
\geometry{left=2.5cm, right=2.5cm, top=2.5cm, bottom=2.5cm}
\linespread{1.5}
\setlength{\headheight}{14pt} 

%=================== 3. 字体设置 ===================
% (1) 中文字体:宋体
\setCJKmainfont[
    Path=fonts/,
    BoldFont=simsun.ttc, 
    AutoFakeBold=true
]{simsun.ttc}

% (2) 英文字体:使用 TeX Gyre Termes(Times 兼容)
\setmainfont{TeX Gyre Termes}

% (3) 页脚专用字体:使用默认字体
\newfontfamily\calibrifont{TeX Gyre Termes}

%=================== 4. 页眉页脚设置 ===================
\pagestyle{fancy}
\fancyhf{} 

% 前言页样式
\fancypagestyle{frontstyle}{
    \fancyhf{}
    \fancyhead[C]{\songti\zihao{5} 最优控制课程设计报告}
    \fancyfoot[C]{\calibrifont\bfseries\zihao{5} \textbf{\thepage}} 
    \renewcommand{\headrulewidth}{0.5pt}
}

% 正文页样式
\fancypagestyle{mainstyle}{
    \fancyhf{}
    \fancyhead[C]{\songti\zihao{5} 最优控制课程设计报告}
    \fancyfoot[C]{\calibrifont\bfseries\zihao{5} \textbf{- \thepage\ -}}
    \renewcommand{\headrulewidth}{0.5pt}
}

%=================== 5. 标题格式 ===================
\ctexset{
    section = {
        name = {第,章},
        number = \arabic{section},
        format = \centering\bfseries\songti\zihao{3}, 
        aftername = \quad,
        beforeskip = 24pt,
        afterskip = 18pt,
    },
    subsection = {
        format = \bfseries\songti\zihao{-3}, 
        indent = 0em,
        beforeskip = 24pt,
        afterskip = 6pt,
    },
    subsubsection = {
        format = \bfseries\songti\zihao{4}, 
        indent = 0em,
        beforeskip = 12pt,
        afterskip = 6pt,
    }
}

%=================== 6. 目录格式 ===================
\titlecontents{section}[0pt]{\addvspace{2pt}\filright}
              {\songti\zihao{5} \thecontentslabel \quad}
              {}{\titlerule*[8pt]{.}\contentspage}
              
\titlecontents{subsection}[2em]{\addvspace{2pt}\filright}
              {\songti\zihao{5} \thecontentslabel\quad}
              {}{\titlerule*[8pt]{.}\contentspage}
              
\titlecontents{subsubsection}[4em]{\addvspace{2pt}\filright}
              {\songti\zihao{5} \thecontentslabel\quad}
              {}{\titlerule*[8pt]{.}\contentspage}

\titlecontents{numberlesssection}[0pt]{\addvspace{2pt}\filright}
              {\songti\zihao{5}}
              {}{\titlerule*[8pt]{.}\contentspage}

%=================== 文档开始 ===================
\begin{document}

% --------- 前言部分 ---------
\pagenumbering{Roman}
\pagestyle{frontstyle}

% 1. 摘要
\phantomsection
\addcontentsline{toc}{section}{摘要}
\begin{center}
    \vspace*{1em}
    {\songti\bfseries\zihao{3} 摘要}
\end{center}
\vspace{1em}

{\songti\zihao{-4}
\todo{【第一组成员】在此处翻译论文的 Abstract 部分。内容应包括:回顾了模型预测控制(MPC)在微型飞行器(MAVs)中的设计与应用,涵盖线性与非线性动力学、状态与输入约束集成、容错设计、强化学习以及负载运输等任务 [cite: 4-8]。}

\vspace{1em}
{\bfseries
    \noindent Abstract:
    \todo{【Group 1】Paste the English abstract here. Use font Times New Roman Bold.}
}}

\clearpage

% 2. 目录
{
    \songti\zihao{5}
    \tableofcontents
}
\clearpage

% --------- 正文部分 ---------
\pagenumbering{arabic}
\pagestyle{mainstyle}

% ===================== 第1章 引言 =====================
\section{引言}
\todo{【第一组成员】翻译 I. INTRODUCTION。}

\todo{重点内容:MAVs(特别是多旋翼)在巡检、监视等领域的广泛应用及其优势 [cite: 10-12]。强调控制器的准确性与鲁棒性是关键 [cite: 13]。引出模型预测控制(MPC)在处理约束、优化轨迹方面的优势 [cite: 24-28]。简述本文结构 [cite: 31-35]。}

% ===================== 第2章 建模 =====================
\section{微型飞行器建模}
\todo{【第一组成员】翻译 II. MODELING OF MICRO AERIAL VEHICLES。}

\subsection{动力学模型}
\todo{请详细推导以下内容:}
\begin{itemize}
    \item \todo{坐标系定义:惯性系 $I$ 与机体系 $B$ 的定义 [cite: 73];}
    \item \todo{螺旋桨推力 $F_T$ 与力矩 $M_i$ 模型 (公式 1) [cite: 77-81];}
    \item \todo{包含气动阻力效应的受力分析 (公式 2) [cite: 83-86];}
    \item \todo{最终的平动与转动动力学方程 (公式 3-6),注意解释 $R_{IB}$ 及外力项 $F_{ext}$ [cite: 87-97]。}
\end{itemize}

\subsection{姿态子系统}
\todo{翻译关于姿态内环控制的假设。通常使用一阶模型近似闭环姿态动力学 [cite: 98-100]。请录入公式 (7),描述 $\phi, \theta$ 的响应特性 [cite: 105-107]。}

% ===================== 第3章 MPC综述 =====================
\section{微型飞行器模型预测控制}
\todo{【第二组成员】翻译 III. MODEL PREDICTIVE CONTROL FOR MAVS 章节开头的概述 [cite: 108-111]。}

\subsection{线性模型预测控制}
\todo{【第二组成员】翻译 A. Linear Model Predictive Control。}

\subsubsection{模型线性化}
\todo{翻译在悬停点附近的线性化过程。定义状态向量 $x$ 和输入 $u$ (公式 8-9) [cite: 115-116]。给出离散状态空间方程 $x_{k+1}=Ax_k+Bu_k+B_dF_{ext,k}$ (公式 11) [cite: 124]。}

\subsubsection{最优控制问题构建}
\todo{重点翻译公式 (12)-(13),即 LMPC 的代价函数 $J = \sum (||x_k-x_{ref,k}||^2_{Q_x} + ||u_k-u_{ref,k}||^2_{R_u})$ 及其约束条件 [cite: 127-133]。}

\subsubsection{扰动观测器}
\todo{翻译基于扩维状态的扰动观测器设计 (公式 16) 及其在实现无静差跟踪中的作用 [cite: 142-151]。}

\subsection{非线性模型预测控制}
\todo{【第三组成员】翻译 B. Nonlinear Model Predictive Control。}

\todo{翻译非线性最优控制问题的积分形式 (公式 18) [cite: 176]。描述状态约束 $x \in \mathcal{X}$ 和输入约束 $u \in \mathcal{U}$。介绍直接法(Direct Methods)和多重射击法(Multiple Shooting)在求解该问题中的应用 [cite: 184-186]。}

\subsection{线性与非线性MPC的比较}
\todo{【第三组成员】翻译 C. Comparison of Linear and Nonlinear MPC。}

\todo{这是重点实验分析部分。请完成:}
\begin{itemize}
    \item \todo{插入 Figure 4:正弦轨迹跟踪对比。分析 NMPC 在剧烈机动下优于 LMPC 的原因 [cite: 260-262]。}
    \item \todo{插入 Figure 5:质量参数不确定下的阶跃响应。分析扰动观测器的必要性 [cite: 264]。}
    \item \todo{插入 Figure 7 或 Table I:关于计算时间与预测时域 $N$ 的关系分析 [cite: 225-236, 287-294]。}
\end{itemize}

\subsection{容错MPC}
\todo{【第四组成员】翻译 D. Fault-Tolerant MPC。}

\todo{描述 MAV 在螺旋桨部分或完全失效(如六旋翼失效3个电机)的情况下,如何利用 MPC 的约束处理能力保持可控性 [cite: 326-338]。}

\subsection{深度强化学习}
\todo{【第四组成员】翻译 E. Deep Reinforcement Learning。}

\todo{探讨 RL 与 MPC 的结合点:1. 利用 RL 学习价值函数作为 MPC 的终端代价 [cite: 342];2. 使用神经网络近似 MPC 策略以加速计算(Policy Search/Compression)[cite: 345-347]。}

\subsection{负载运输}
\todo{【第五组成员】翻译 F. Load Transportation。}

\todo{翻译关于吊挂负载(Cable-suspended load)的控制挑战。MPC 如何处理负载摆动以及多机协同运输问题 [cite: 350-358]。}

\subsection{物理交互}
\todo{【第五组成员】翻译 G. Physical Interaction。}

\todo{描述混合系统(Hybrid Systems)建模在物理交互(如接触墙面、开关门)中的应用。提及图 8 (Figure 8) 的交互动力学概念 [cite: 359-382]。}


% ===================== 第5章 结论 =====================
\section{结论}
\todo{【第五组成员】翻译 V. CONCLUSIONS。}

\todo{总结全文:MPC在MAV领域的现状(线性/非线性、容错、RL结合)及未来展望 [cite: 395-398]。}

% --------- 附录部分 ---------
\clearpage
\appendix

% 使用无编号 section*,并手动添加目录,避免出现 "A 附录"
\section*{附录\quad 组员分工与心得体会}
\addcontentsline{toc}{section}{附录\quad 组员分工与心得体会}

{\songti\zihao{-4}
    \begin{itemize}
        \item \textbf{组员1}:负责摘要、引言及建模部分(第1-2章)。心得:通过推导公式(1)-(6),深入理解了多旋翼飞行器的动力学特性...
        \item \textbf{组员2}:负责线性MPC理论推导(第3.1节)。心得:掌握了利用泰勒展开进行模型线形化的方法...
        \item \textbf{组员3}:负责非线性MPC及对比实验分析(第3.2-3.3节)。心得:...
        \item \textbf{组员4}:负责容错控制与深度强化学习部分(第3.4-3.5节)。心得:...
        \item \textbf{组员5}:负责特殊应用场景、开源资源及结论(第3.6-5章)。心得:...
    \end{itemize}
}

% --------- 参考文献 ---------
\clearpage
\phantomsection
\addcontentsline{toc}{section}{参考文献}

\renewcommand{\refname}{\centering\songti\zihao{3}\bfseries 参考文献}

{
    \songti\zihao{-4}
    \bibliographystyle{ieeetr} % IEEE 风格,按引用顺序排序
    \bibliography{ref} % 确保根目录下有 ref.bib 文件
}
\end{document}